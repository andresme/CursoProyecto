\documentclass[12pt]{article}

\usepackage{graphicx}
\usepackage[hidelinks]{hyperref}


\newcommand{\specialcell}[2][c]{%
  \begin{tabular}[#1]{@{}c@{}}#2\end{tabular}}



\title{\begin{flushright}\textbf{SAD Iteraci\'on 1} \\[0.7in] 
		Proyecto: \\[0.2in]
		\textbf{Aplicaci\'on M\'ovil para la cl\'inica de Salud del Tecnol\'ogico de Costa Rica}\\[0.7in]
		Versi\'on 1.0 \\[0.7in]
		Preparado por: \\[0.2in]
		\textbf{Irene Gamboa Padilla.\\
		Andr\'es Morales Esquivel.} \\[0.7in]
		18 de Marzo de 2013
		\end{flushright}}
\author{}
\date{}


\begin{document}

\maketitle
\newpage
\renewcommand{\contentsname}{Tabla de Contenido}
{\footnotesize
	\tableofcontents
}
\section{Historia de Revisiones}

\begin{center}
	\begin{tabular}{|c|c|c|c|}
	\hline
		\textbf{Nombre} & \textbf{Fecha} & \textbf{Razones para los cambios} & \textbf{Versi\'on}\\
	\hline
		SAD & 18/03/2013 & Inicio del documento & 1.0\\
	\hline
	\end{tabular}
\end{center}

\newpage
\section{Introducci\'on}

\subsection{Prop\'osito}
Este documento provee un resumen de toda la arquitectura del sistema, usando varias vistas de arquitectura que describen diferentes partes del sistema. Es realizado para capturar varias decisiones importantes del sistema que se va a desarrollar.

\subsection{Alcance}
El proyecto en general es afectado por este documento, el dise\~no de este sistema es afectado por este documento ya que este contiene las principales vistas que se necesitan para que los desarrolladores del sistema tengan una noci\'on de como se va a implementar el mismo.

\subsection{Definiciones, Acr\'onimos, Abreviaciones}
\begin{itemize}
	\item{Legacy Code: C\'odigo, no soportado pero que se usa para portabilidad.}
	\item{Portable: Que pueda correr en varios dispositivos sin cambiar el c\'odigo fuente.}
	\item{Libreria de Soporte de Google: Libreria de soporte para versiones anteriores de Android}
\end{itemize}

\subsection{Referencias}

\subsection{Resumen}
Este documento, como ya se mencion\'o contiene las vistas principales del sistema para el desarrollo del proyecto, y estar\'a organizado por Vistas.


\section{Representaci\'on de la Arquitectura}
Las vistas que se utilizar\'an para representar el producto de este proyecto son: 
\begin{itemize}
	\item{Vista de Casos de Uso: Contiene el diagrama de casos de Uso.}
	\item{Vista L\'ogica: Contiene el diagrama de paquetes.}
	\item{Vista de Procesos: Informaci\'on de cada proceso que se va a realizar.}
	\item{Vista de Despligue: Informaci\'on del despliegue de la aplicaci\'on.}
	\item{Vista de Implemetaci\'on: Interacci\'on con otros subsistemas y capas que utilizar\'a el proyecto.}
\end{itemize}

\section{Metas y Restricciones de la Arquitectura}
Uno de los objetivos de esta arquitectura es brindar informaci\'on rapida a los usuarios, y debe ser portable, debido a que ser\'a usada en celulares y dispositivos portables. En cuanto a la implementaci\'on se va a soportar "legacy code" mediante la libreria de soporte de Google.

\section{Vista de Casos de Uso}
\includegraphics[width=\linewidth]{"../Documentos/Diagramas/casos de uso"}\\
	
El diagrama anterior indica los casos de uso a implementar durante la primera iteraci\'on del proyecto, los casos de uso anteriores ser\'an desarrollados m\'as adelante en este documento.
	


\section{Vista L\'ogica}
En esta secci\'on se mostrar\'a la distribuci\'on de paquetes y clases que tendr\'a la primera iteraci\'on del proyecto mediante el siguiente diagrama:\\

\includegraphics[width=\linewidth]{"../Documentos/Diagramas/Logical"}

\subsection{Resumen}
Este diagrama muestra los paquetes y divisiones logicas de la aplicaci\'on a desarrollar en este proyecto, explicando como va a ser la arquitectura logica de la aplicaci\'on.

\subsection{Arquitectura de Paquetes Importantes de Dise\~no}

\subsection{DataAccess}
El paquete de DataAccess contiene dos clases, un DatabaseHelper, que es el encargado de manejar todo el acceso a la base de datos, y una clase que sera la encargada de almacenar las funciones que utilizen los Store Procedures de la base de datos.\\
\includegraphics[scale=0.5]{"../Documentos/Diagramas/DA"}

\subsection{Activities}
El paquete de Activities, contiene todos los Activities, son clases que utiliza Android para cada acci\'on que se desea hacer.\\
\includegraphics[scale=0.5]{"../Documentos/Diagramas/Act"}

\subsection{DataStructures}
El paquete de Data Structures, es el que contiene todas las estructuras de datos que van a ser utilizadas por todas las Activities.\\
\includegraphics[scale=0.5]{"../Documentos/Diagramas/DS"}

\subsection{Casos de Uso}

\subsubsection{Mapa}
Este caso de uso sirve para que el usuario vea un mapa del campus, con la ubicaci\'on de la cl\'inica de salud del TEC.

\paragraph{Diagrama de Caso de Uso}\ \\
\includegraphics[scale=0.6]{"../Documentos/Diagramas/CU1_"}

\paragraph{Descripci\'on}\ \\
Las acciones de este caso de uso son 3:
\begin{itemize}
	\item{Ver Mapa: Mostrar el mapa en la pantalla del dispositivo.}
	\item{Mover Mapa: Mover el mapa por la pantalla para ir al objetivo deseado.}
	\item{Zoom: Acercar o alejar el mapa.}
\end{itemize}

\subsubsection{Servicios Disponibles}
Este caso de uso muestra los servicios disponibles en la cl\'inica de salud, con informaci\'on sobre el servicio.

\paragraph{Diagrama de Caso de Uso}\ \\
\includegraphics[scale=0.6]{"../Documentos/Diagramas/CU2_"}

\paragraph{Descripci\'on}\ \\
Las acciones de este caso de uso son 2:
\begin{itemize}
	\item{Ver Servicios: Mostrar los servicios en modo de lista en la pantalla.}
	\item{Ver Informaci\'on: Desplegar informaci\'on basica del servicio que se desea.}
\end{itemize}

\section{Vista de Procesos}
En esta iteraci\'on del proyecto los procesos principales ser\'an:

\begin{itemize}
	\item{Obtener la informaci\'on de los servicios de la cl\'inica de salud.}
	\item{Actualizar la imagen del mapa cada vez que el usuario lo requiera mediante el uso del zoom o del movimiento del mapa.}
\end{itemize}

\section{Vista de Despliegue}
Este proyecto ser\'a desarrollado para Android, por lo tanto lo necesario para poder correr el producto desarrollado por este proyecto es un dispositivo movil Android con la version 2.3.3 o superior.

\section{Vista de Implementaci\'on}


\subsection{Resumen}
Este proyecto no requiere el uso de subsistemas o sistemas externos, todos los paquetes y modelos significantes fueron explicados anteriormente.

\subsection{Capas}
Las capas para una aplicaci\'on desarrollada para Android es muy similar al de MVC, hay un Model que maneja los datos, hay un View que es la interfaz, en este caso son archivos XML con la descripcion de la interfaz, y un Controller que es quien maneja todo.


\section{Tama\~no y Desempe\~no}
El tama\~no de la aplicaci\'on no ser\'a mayor a 10mbs ya instalado, el desempe\~no se espera que sea de respuesta inmediata ya que no requiere hacer operaciones complicadas, esto para que corra en dispositivos con bajo poder de procesamiento.

\section{Calidad}
La arquitectura de este proyecto no es muy compleja, por lo que permite un desarrollo rapido y tambien permite que corra en dispositivos moviles que no tienen mucha capacidad de procesamiento.

\end{document}
