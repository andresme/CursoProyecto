\documentclass[12pt]{article}

\usepackage{graphicx}
\usepackage{lscape}
\usepackage{pgfgantt}
\usepackage[hidelinks]{hyperref}


\newcommand{\specialcell}[2][c]{%
  \begin{tabular}[#1]{@{}c@{}}#2\end{tabular}}



\title{\begin{flushright}\textbf{SAD Iteraci\'on 1} \\[0.7in] 
		Proyecto: \\[0.2in]
		\textbf{Aplicaci\'on M\'ovil para la cl\'inica de Sal\'ud del Tecnol\'ogico de Costa Rica}\\[0.7in]
		Versi\'on 1.0 \\[0.7in]
		Preparado por: \\[0.2in]
		\textbf{Irene Gamboa Padilla.\\
		Andr\'es Morales Esquivel.} \\[0.7in]
		11 de Marzo de 2013
		\end{flushright}}
\author{}
\date{}


\begin{document}

\maketitle
\newpage
\renewcommand{\contentsname}{Tabla de Contenido}
{\footnotesize
	\tableofcontents
}
\section{Historia de Revisiones}

\begin{center}
	\begin{tabular}{|c|c|c|c|}
	\hline
		\textbf{Nombre} & \textbf{Fecha} & \textbf{Razones para los cambios} & \textbf{Versi\'on}\\
	\hline
		ERS & 05/03/2013 & Inicio del documento & 1.0\\
	\hline
	\end{tabular}
\end{center}

\newpage

\section{Introducci\'on}

\subsection{Prop\'osito del Documento ERS iteraci\'on 1}

El prop\'osito general de este documento es darle a conocer al usuario la forma correcta de trabajar en el sistema a desarrollar. Le mostramos como se ha desarrollado cada una de las \'areas importantes en el desarrollo de nuestro sistema.

\subsection{Descripci\'on del Problema}

El problema que este sistema pretende solucionar es la falta de informaci\'on de la comunidad institucional con respecto a la cl\'inica de sal\'ud.

\subsection{Lista de Problemas Detectados}
\begin{itemize}
	\item{La pagina institucional no se encuentra actualizada con informaci\'on de este departamento.}
\end{itemize}

\subsection{Lista de Fortalezas Detectadas}
\begin{itemize}
	\item{El departamento cuenta con toda la informaci\'on necesaria para la base de datos del sistema.}
\end{itemize}

\subsection{Objetivos del Sistema}

\subsubsection{Objetivo General}

\begin{itemize}
\item{Objetivo General}
\\[0.1in]
Brindar informaci\'on acerca del departamento de Salud a la comunidad del Instituto Tecnol\'ogico de Costa Rica mediante el desarrollo de una aplicaci\'on m\'ovil para telefonos Android.

\subsubsection{Objetivos Especificos}

\item{Objetivos Espec\'ificos}
\begin{itemize}
	\item{Informar sobre horarios de atenci\'on.}
	\item{Brindar un mapa con ubicaciones importantes.}
	\item{Permitir a los usuarios ingresar/modificar sus citas para generar las notificaciones.}
	\item{Notificar a los usuarios sobre sus citas.}
	\item{Permitir la navegaci\'on a sitios web de inter\'es de la cl\'inica de salud.}
	\item{Dar informaci\'on de contacto del personal del Departamento de Trabajo Social y Salud.}
\end{itemize}
\end{itemize}

\subsubsection{Criterios de \'Exito}
Desarrollar la aplicaci\'on para el sistema operativo Android, que permita a los funcionarios ver informaci\'on de la cl\'inica de Salud, mapas, horarios de atenci\'on, informaci\'on sobre los funcionarios: nombre, correo electr\'onico, tel\'efono. Tambi\'en permitir al usuario ingresar sus citas para generar notificaciones cuando su cita se aproxima.

\subsection{Perspectiva del Producto por Desarrollar}
El producto a desarrollar ser\'a una aplicaci\'on para Android, el cual ayudar\'a a la comunidad estudiantil del Instituto Tecnol\'ogico de Costa Rica a adquirir la informaci\'on necesaria sobre el departamento de salud(cl\'inica) que este brinda.

\subsection{Suposiciones y Dependencias}

\begin{itemize}

\item{Supociciones}
\begin{itemize}
	\item{El sistema operativo de este producto ser\'a Android.}
	\item{La versi\'on de Android en la que se desarrollar\'a el sistema ser\'a 2.3.}
\end{itemize}

\item{Dependencias}
\begin{itemize}
	\item{El Depto. de Trabajo Social y Salud brindar\'a la informaci\'on necesaria para la aplicaci\'on.}
\end{itemize}

\end{itemize}

\subsection{Alcances del Sistema}

\begin{itemize}
	\item{Horarios.}
	\item{Ubicaci\'on.}
	\item{Informaci\'on de Contacto.}
	\begin{itemize}
		\item{Opci\'on para marcado de alg\'un contacto.}
		\item{Opci\'on para enviar correo electronico.}
	\end{itemize}
	\item{Link a sitios web de inter\'es.}
	\item{Agregar citas a la base de datos.}
	\item{Configuraci\'on de tiempo de las notificaciones.}
\end{itemize}

\subsection{Limitaciones o Restricciones}

\begin{itemize}
	\item{Conectividad con otros sistemas para una entrada de noticias a la aplicaci\'on.}
	\item{Notificaciones del Depto. de Trabajo Social y Salud acerca de cambios de citas.}
	\item{Solicitud de citas atrav\'ez de la aplicaci\'on.}
\end{itemize}

\subsection{StakeHolders y sus Necesidades}

\begin{small}
\begin{tabular}{|c|c|c|c|c|}
\hline
	Interesado & Necesidad\\
\hline
	\specialcell{Depto.\\Trabajo Social y Salud} & Divulgar informaci\'on acerca del departamento\\
\hline
	\specialcell{Comunidad\\Insititucional} & \specialcell{Tener f\'acil acceso a informaci\'on sobre el departamento de salud.} \\
\hline
\end{tabular}
\end{small}

\section{Requerimientos Funcionales}

\subsection{Contexto del Sistema}

\subsubsection{Diagrama de Contexto}

\includegraphics[width=\linewidth]{"../Documentos/Diagramas/diagrama contexto"}

\subsubsection{Modelo de Dominio del Sistema}

\includegraphics[width=\linewidth]{"../Documentos/Diagramas/modelo_dominio"}

\subsubsection{Descripci\'on Modelo de Dominio}

Como se puede notar en el modelo de dominio, el Departamento de Salud, especificamente la cl\'inica, tiene 3 relaciones, una con el "Usuario", esta relaci\'on es que la cl\'inica atiende a un usuario (de la comunidad institucional), una relaci\'on con el "Catalogo de Servicios" esta relaci\'on significa que la cl\'inica tiene varios servicios a ofrecer, los cuales se describen en "Servicio" de esta forma los usuarios tienen una manera de ver los servicios ofrecidos por la cl\'inica, por \'ultimo la cl\'inica tambi\'en tiene un "Libro de Contactos", el cual posee "Contacto" esto para tener un lugar en donde se puedan ver los principales contactos de este departamento en caso de necesitar contactar a alguno de los funcionarios.\\
El "Usuario" esta relacionado con "Agenda", esto es que cada usuario(solo uno ya que la aplicaci\'on es de uso personal), tiene su agenda con sus citas en la cl\'inica.

\subsubsection{Diagrama Casos de Uso (Iter. 1)}

\includegraphics[width=\linewidth]{"../Documentos/Diagramas/casos de uso"}

\newpage
\subsection{Descripci\'on Detallada de Casos de Uso}

\subsubsection{Caso de Uso 1 - Ver Mapa}
\paragraph{Texto del Caso de Uso}\ \\

Este caso de uso permite al usuario ver la ubicaci\'on de la cl\'inica de salud dentro del campus del Instituto Tecnol\'ogico de Costa Rica, y tambi\'en permite la funcionalidad de acercamiento del mapa (hacia afuera y hacia dentro) y tambi\'en la funcionalidad de navegabilidad (mover el mapa), esto con una mapa propio del departamento de salud.

\paragraph{Pantallas}\ \\

\includegraphics[scale=0.5]{"../Documentos/Diagramas/MockupMapa"}

\newpage
\paragraph{Diagrama de Actividades}\ \\

\includegraphics[width=\linewidth]{"../Documentos/Diagramas/CU1_Actividades"}


\paragraph{Digarama de Estados}\ \\

\includegraphics[width=\linewidth]{"../Documentos/Diagramas/CU1_Estados"}

\newpage
\paragraph{Diagrama de Secuencia del Sistema}\ \\

\includegraphics[width=\linewidth]{"../Documentos/Diagramas/CU1_SecuenciaSistema"}

\paragraph{Contratos de Operaciones}
No aplica para este caso de uso.

\paragraph{Casos de Prueba del Caso de Uso}
\begin{itemize}
	\item{ScrollTest: Probar que el scroll funcione correctamente.}
	\item{ZoomTest: Probar que el zoom funcione correctamente.}
\end{itemize}

\newpage
\subsubsection{Caso de Uso 2 - Ver Servicios Disponibles}
\paragraph{Texto del Caso de Uso}\ \\

Este caso de uso permite al usuario ver la informaci\'on de los servicios brindados por la cl\'inica de salud del Instituto Tecnol\'ogico de Costa Rica. A su vez permite el enlace de estos servicios con su sitio web.

\paragraph{Pantallas}\ \\

\includegraphics[scale=0.5]{"../Documentos/Diagramas/MockupServicios"}

\newpage
\paragraph{Diagrama de Actividades}\ \\

\includegraphics[width=\linewidth]{"../Documentos/Diagramas/CU2_Actividades"}

\paragraph{Digarama de Estados}\ \\

\includegraphics[width=\linewidth]{"../Documentos/Diagramas/CU2_Estados"}

\newpage
\paragraph{Diagrama de Secuencia del Sistema}\ \\

\includegraphics[width=\linewidth]{"../Documentos/Diagramas/CU2_SecuenciaSistema"}
\paragraph{Contratos de Operaciones}
No aplica para este caso de uso.
\paragraph{Casos de Prueba del Caso de Uso}
\begin{itemize}
	\item{DesplegarServiciosTest: Probar que la informaci\'on de los servicios se desplegue }
	\item{ConectarSitioWebTest: Probar que haga correctamente el enlance al sitio web.}
\end{itemize}

\newpage
\section{Requerimientos No Funcionales}

\subsection{Producto}
\subsubsection{Eficiencia}
Se espera que las notificaciones tengan el efecto de avisar una cita N minutos antes de la misma, el usuario especifica esos N minutos.\\ 
En el mapa el efecto de scroll y zoom, se espera sea inmediato (menor a un segundo).

\subsubsection{Interfaz de Usuario}
Con la Interfaz de usuario se espera una f\'acil navegabilidad entra las diferentes pantallas de la aplicaci\'on.


\subsection{Seguridad}
Debido a que no hay usuarios distintos el manejo de seguridad es inexistente en esta aplicaci\'on. El usuario no especifica ningun tipo de seguridad para la informaci\'on del departamento.

\subsection{Organizacionales}
\subsubsection{Documentaci\'on}
Toda Documentaci\'on se debe entregar por correo, debe estar en espa\~nol.

\subsubsection{Entregas}
Se realizara una entrega de toda la documentaci\'on y un sistema funcional con al menos 2 casos de uso durante la duraci\'on de este proyecto.

\subsubsection{Implementaci\'on}
Este es un Sistema m\'ovil, por lo tanto la implementaci\'on de este sistema es por parte de los usuarios finales en cada dispositivo m\'ovil por medio de una descarga de alg\'un link en Internet.


\newpage
\section{Ap\'endice}
\begin{landscape}
\subsection{Plan del Proyecto}
\begin{center}

\begin{ganttchart}[y unit title=0.4cm,
y unit chart=0.46cm,
vgrid,hgrid, 
title label anchor/.style={below=-1.6ex},
title left shift=.05,
title right shift=-.05,
title height=1,
bar/.style={fill=gray!50},
incomplete/.style={fill=green!50},
progress label text={},
bar height=0.8,
group right shift=0,
group top shift=.6,
group height=.5,
group peaks={}{}{.2}]{28}
%labels
\gantttitle{Semanas}{28} \\
\gantttitle{1}{2} 
\gantttitle{2}{2} 
\gantttitle{3}{2} 
\gantttitle{4}{2} 
\gantttitle{5}{2} 
\gantttitle{6}{2}
\gantttitle{7}{2}
\gantttitle{8}{2}
\gantttitle{9}{2}
\gantttitle{10}{2}
\gantttitle{11}{2}
\gantttitle{12}{2}
\gantttitle{13}{2} 
\gantttitle{14}{2} \\
%tasks
\ganttbar[progress=0]{Presentar Propuesta}{1}{2} \\
\ganttbar[progress=0]{Definir Visi\'on}{3}{4} \\
\ganttbar[progress=0]{Prototipo Aplicaci\'on}{5}{5} \\
\ganttbar[progress=0]{Presentar Prototipo}{6}{6} \\

\ganttbar[progress=0]{Creaci\'on ERS Iter. 1}{7}{8} \\
\ganttbar[progress=0]{Creaci\'on SAT Iter. 1}{9}{10} \\
\ganttbar[progress=0]{Creaci\'on Plan Pruebas Iter. 1}{7}{8} \\
\ganttbar[progress=0]{Desarrollo 2 casos de uso}{7}{12} \\
\ganttbar[progress=0]{Creaci\'on Manual de Usuario}{11}{11} \\
\ganttbar[progress=0]{Presentaci\'on Iter. 1}{11}{11} \\

\ganttbar[progress=0]{Creaci\'on ERS Iter. 2}{13}{14} \\
\ganttbar[progress=0]{Creaci\'on SAT Iter. 2}{15}{16} \\
\ganttbar[progress=0]{Creaci\'on Plan Pruebas Iter. 2}{13}{14} \\
\ganttbar[progress=0]{Desarrollo 2 casos de uso}{13}{18} \\
\ganttbar[progress=0]{Creaci\'on Manual de Usuario}{17}{17} \\
\ganttbar[progress=0]{Presentaci\'on Iter. 2}{17}{17} \\

\ganttbar[progress=0]{Creaci\'on ERS Iter. 3}{19}{20} \\
\ganttbar[progress=0]{Creaci\'on SAT Iter. 3}{21}{22} \\
\ganttbar[progress=0]{Creaci\'on Plan Pruebas Iter. 3}{19}{20} \\
\ganttbar[progress=0]{Desarrollo 2 casos de uso}{19}{24} \\
\ganttbar[progress=0]{Creaci\'on Manual de Usuario}{23}{23} \\
\ganttbar[progress=0]{Presentaci\'on Iter. 3}{23}{23} \\

\ganttbar[progress=0]{Plan e Informe del Sistema}{25}{26} \\
\ganttbar[progress=0]{Manual del sistema}{27}{28} 


\end{ganttchart}
\end{center}

\end{landscape}
\subsection{Glosario}
\begin{itemize}
	\item{Agenda: Una lista de actividades, en este caso citas con la cl\'inica de salud.}
	\item{Android: Es un sistema operativo basado en Linux, dise\~nado principalmente para m\'oviles con pantalla t\'actil como tel\'efonos inteligentes o tabletas.}
	\item{Contacto: Informaci\'on sobre una persona.}
	\item{Libro de Contactos: Pantalla donde se podr\'an ver todos los contactos disponibles en la cl\'inica.}
	\item{Notificaciones: Alarma con texto de un dispositivo con Android.}
	\item{Scroll: Termino utilizado para referirse a arrastrar una imagen por la pantalla.}
	\item{StakeHolders: Interesados.}
	\item{Zoom: Termino utilizado para referirse a agrandar o hacer m\'as peque\~na una imagen.}
\end{itemize}

\subsection{Estandares de Programaci\'on}

El estandar de progrmaci\'on que se utilizar\'a en este proyecto es el estadar propuesto por google, en la siguiente pagina: \url{http://source.android.com/source/code-style.html}.

\end{document}
