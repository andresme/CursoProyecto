\documentclass[11pt]{article}


\newcommand{\specialcell}[2][l]{%
  \begin{tabular}[#1]{@{}l@{}}#2\end{tabular}}

\title{\begin{flushright}\textbf{Visi\'on y Alcance} \\[0.7in] 
		Proyecto: \\[0.2in]
		\textbf{Aplicaci\'on M\'ovil para Depto. Salud}\\[0.7in]
		Versi\'on 1.0 \\[0.7in]
		Preparado por: \\[0.2in]
		\textbf{Irene Gamboa Padilla.\\
		Andr\'es Morales Esquivel.} \\[0.7in]
		25 de febrero de 2013
		\end{flushright}}
\author{}
\date{}
\begin{document}

\maketitle
\newpage
\renewcommand{\contentsname}{Tabla de Contenido}
\tableofcontents


\section{Historia de Revisiones}

\begin{tabular}{|c|c|c|c|}
\hline
	\textbf{Nombre} & \textbf{Fecha} & \textbf{Razones para los cambios} & \textbf{Versi\'on}\\
\hline
	Visi\'on del Proyecto & 22/02/2013 & Inicio del documento & 1.0\\
\hline
	 &  &  & \\
\hline
\end{tabular}

\newpage

\section{Introducci\'on}



\subsection{Antecedentes del Problema}

Este proyecto pretende dar soluci\'on a un problema del departamento de salud, el cual es brindar informaci\'on b\'asica a los estudiantes sobre el mismo.

\subsection{Oportunidad del producto por desarrollar}

Mediante este producto el cual se desarrollar\'a para smartphones con la capacidad de correr el sistema operativo Android, se brindar\'ia esta informaci\'on a los estudiantes con dicho dispositivo, el cual cada vez es m\'as com\'un entre los estudiantes.

\subsection{Estatuto del Problema}

\begin{center}\begin{tabular}{|l|l|}
\hline
	Problema: & \specialcell{Brindar informaci\'on acerca del departamento\\ de salud a los estudiantes del Tecnol\'ogico.}\\
\hline
	Afectados: & Estudiantes y departamento de salud.\\
\hline
	Impacto: & \specialcell{Divulgar informaci\'on del departamento de salud\\ mediante el uso de tecnol\'ogia.}\\
\hline
	Soluci\'on & Divulgar informaci\'on a los estudiantes del tecnol\'ogico.\\
\hline
\end{tabular}\end{center}

\subsection{Objetivos del Sistema}

\begin{itemize}
\item{Objetivo General}
\\[0.1in]
Brindar informaci\'on acerca del departamento de salud a los estudiantes del Instituto Tecnol\'ogico de Costa Rica mediante el desarrollo de una aplicaci\'on m\'ovil para telefonos Android.

\item{Objetivos Espec\'ificos}
\begin{itemize}
	\item{Informar sobre horarios de atenci\'on.}
	\item{Brindar un Mapa con ubicaciones importantes.}
	\item{Recordar a los estudiantes sobre sus citas.}
	\item{Dar informaci\'on de contacto del personal del departamento de salud.}
\end{itemize}
\end{itemize}

\subsection{Criterios de \'Exito}

Desarrollar la aplicaci\'on para el sistema operativo Android con una interfaz de usuario amigable, para que los estudiantes se sientan comodos al usarla y que esta aplicaci\'on sea usada por los estudiantes para obtener informaci\'on sobre este departamento.

\subsection{Necesidades del cliente}

\subsection{Riesgos del Negocio}

\section{Visi\'on de la Soluci\'on}

\subsection{Estatuto de la visi\'on}

\subsection{Características principales}

\subsection{Suposiciones y dependencias}

\subsection{Costo y Precio}

\subsection{Licencias e Instalaci\'on}

Debido a que se utilizara unicamente software libre para este proyecto, ser\'a innecesaria cualquier tipo de software.\\
La instalaci\'on de este sistema se hara mediante un link a una descarga la cual ser\'a instalable en los dispositivos con sistema operativo Android.

\section{Alcances y Limitaciones}

\subsection{Alcance de la Versi\'on Inicial}

\subsection{Alcance de las Versiones Siguientes}

\subsection{Limitaciones y Exclusiones}

\section{Contexto del Sistema}

\subsection{Diagrama de Contexto del Sistema}

\subsection{Perfiles de los Stakeholder}

\subsection{Prioridades del Proyecto}

\subsection{Ambiente Operativo}

\section{Calendario}


\end{document}
